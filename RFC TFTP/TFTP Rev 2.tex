\documentclass[12pt]{article}
\title{THE TFTP PROTOCOL (REVISION 2)}
\makeindex

\begin{document}
\maketitle
\clearpage
\tableofcontents
\clearpage

\section{The protocol}
\subsection{Purpose}
\textbf{TFTP} is a simple protocol to transfer files, and therefore was named the \textbf{Trivial File Transfer Protocol} or \textbf{TFTP}. It has been implemented on top of the \textbf{Internet User Datagram protocol} (\textbf{UDP} or \textbf{Datagram}) so it may be used to move files between machines on different networks implementing \textbf{UDP}. It is designed to be small and easy to implement. Therefore, it lacks most of the features of a regular \textbf{FTP}. The only thing it can do is read and write files (or mail) from/to a remote server. It cannot list directories, and currently has no provisions for user authentication. In common with other Internet protocols, it passes 8 bit bytes of data.

\subsection{Overview of the Protocol}
Any transfer begins with a request to read or write a file, which also serves to request a connection. If the server grants the request, the connection is opened and the file is sent in fixed length blocks of 512 bytes.  Each data packet contains one block of data, and must be acknowledged by an acknowledgment packet before the next packet can be sent. A data packet of less than 512 bytes signals termination of a transfer. If a packet gets lost in the network, the intended recipient will timeout and may retransmit his last packet (which may be data or an acknowledgment), thus causing the sender of the lost packet to retransmit that lost packet. The sender has to keep just one packet on hand for retransmission, since the lock step acknowledgment guarantees that all older packets have been received. Notice that both machines involved in a transfer are considered senders and receivers. One sends data and receives acknowledgments, the other sends acknowledgments and receives data.\\
Most errors cause termination of the connection. An error is signaled by sending an \verb|ERROR| packet. This packet is not acknowledged, and not retransmitted (i.e., a TFTP server or user may terminate after sending an   error message), so the other end of the connection may not get it. Therefore timeouts are used to detect such a termination when the \verb|ERROR| packet has been lost. Errors are caused by three types of events: not being able to satisfy the request (e.g., file not found, access violation, or no such user), receiving a packet which cannot be explained by a delay or duplication in the network (e.g., an incorrectly formed packet), and losing access to a necessary resource (e.g., disk full or access denied during a transfer). TFTP recognizes only one error condition that does not cause termination, the source port of a received packet being incorrect. In this case, an \verb|ERROR| packet is sent to the originating host. This protocol is very restrictive, in order to simplify implementation. For example, the fixed length blocks make allocation straight forward, and the lock step acknowledgement provides flow control and eliminates the need to reorder incoming data packets.

\subsection{Relation to other Protocols}
As mentioned TFTP is designed to be implemented on top of the Datagram protocol (UDP). Since Datagram is implemented on the Internet protocol, packets will have an Internet header, a Datagram header, and a TFTP header.  Additionally, the packets may have a header (LNI, ARPA header, etc.)  to allow them through the local transport medium. As shown in Figure 3-1, the order of the contents of a packet will be: local medium header, if used, Internet header, Datagram header, TFTP header, followed by the remainder of the TFTP packet. (This may or may not be data depending on the type of packet as specified in the TFTP header.)  TFTP does not specify any of the values in the Internet header. On the other hand, the source and destination port fields of the Datagram header (its format is given in the appendix) are used by TFTP and the length field reflects the size of the TFTP packet. The transfer identifiers (\verb|TID|'s) used by
TFTP are passed to the Datagram layer to be used as ports; therefore they must be between 0 and 65,535. The initialization of \verb|TID|'s is discussed in the section on initial connection protocol. The  TFTP header consists of a 2 byte \verb|opcode| field which indicates the packet's type (e.g., \verb|DATA|, \verb|ERROR|, etc.)  These opcodes and  the formats of  the various types of packets are discussed further in the section on TFTP packets.\\

\begin{center}
Figure 3-1: Order of Headers
\end{center}

\subsection{Initial Connection Protocol}
A transfer is established by sending a request (\verb|WRQ| to write onto a foreign file system, or \verb|RRQ| to read from it), and receiving a positive reply, an acknowledgment packet for write, or the first data packet for read. In general an acknowledgment packet will contain the block number of the data packet being acknowledged. Each data packet has associated with it a block number; block numbers are consecutive and begin with one. Since the positive response to a write request is an acknowledgment packet, in this special case the block number will be zero. (Normally, since an acknowledgment packet is acknowledging a data packet, the acknowledgment packet will contain the block number of the data packet being acknowledged.) If the reply is an   \verb|ERROR| packet, then the request has been denied.\\

In order to create a connection, each end of the connection chooses a \verb|TID| for itself, to be used for the duration of that connection. The \verb|TID|'s chosen for a connection should be randomly chosen, so that the probability that the same number is chosen twice in immediate succession is very low.  Every packet has associated with it the two \verb|TID|'s of the ends of the connection, the source \verb|TID| and the destination \verb|TID|.  These \verb|TID|'s are handed to the supporting UDP (or
other datagram protocol) as the source and destination ports. A requesting host chooses its source \verb|TID| as described above, and sends its initial request to the known \verb|TID| 69 decimal (105 octal) on the serving host. The response to the request, under normal operation, uses a \verb|TID| chosen by the server as its source \verb|TID| and the \verb|TID| chosen for the previous message by the requestor as its destination \verb|TID|. The two chosen \verb|TID|'s are then used for the remainder of the transfer.\\

As an example, the following shows the steps used to establish a connection to write a file. Note that \verb|WRQ|, \verb|ACK|, and \verb|DATA| are the names of the write request, acknowledgment, and data types of packets respectively. The appendix contains a similar example for reading a file.
\begin{itemize}
\item[1] Host A sends  a  "\verb|WRQ|"  to  host  B  with  source=  A's  \verb|TID|, destination= 69.
\item[2] Host  B  sends  a "\verb|ACK|" (with block number= 0) to host A with source= B's \verb|TID|, destination= A's \verb|TID|.
\end{itemize}
   
At this point the connection has been established and the first data packet can be sent by Host A with a sequence number of 1. In the next step, and in all succeeding steps, the hosts should make sure that the source \verb|TID| matches the value that was agreed on in steps 1 and 2. If a source \verb|TID| does not match, the packet should be discarded as erroneously sent from somewhere else. An   \verb|ERROR| packet should be sent to the source of the incorrect packet, while not disturbing the transfer.  This can be done only if the TFTP in fact receives a packet with an incorrect \verb|TID|.  If the supporting protocols do not allow it, this particular   \verb|ERROR| condition will not arise.\\

The following example demonstrates a correct operation of the protocol in which the above situation can occur. Host A sends a request to host B. Somewhere in the network, the request packet is duplicated, and as a result two acknowledgments are returned to host A, with different \verb|TID|'s chosen on host B in response to the two requests.  When the first response arrives, host A continues the connection. When the second response to the request arrives, it should be rejected, but there is no reason to terminate the first connection. Therefore, if different \verb|TID|'s are chosen for the two connections on host B and host A checks the source \verb|TID|'s of the messages it receives, the first connection can be maintained while the second is rejected by returning an   \verb|ERROR| packet.

\subsection{TFTP Packets}
TFTP supports five types of packets, all of which have been mentioned above:\\

\begin{tabular}{ |c|c| }
  \hline
  Opcode	&	Operation\\
  \hline
  1			&	Read request (\verb|RRQ|)\\
  \hline
  2		 	&	Write request (\verb|WRQ|)\\
  \hline
  3			&	Data (\verb|DATA|)\\
  \hline
  4			&	Acknowledgment (\verb|ACK|)\\
  \hline
  5		 	&	Error (\verb|ERROR|)\\
  \hline
  \end{tabular}\\

The TFTP header of a packet contains the \verb|opcode| associated with that packet.
\begin{center}
Figure 5-1: \verb|RRQ|/\verb|WRQ| packet
\end{center}

\verb|RRQ| and \verb|WRQ| packets (\verb|opcode|s 1 and 2 respectively) have the format shown in Figure 5-1. The file name is a sequence of bytes in netascii terminated by a zero byte. The mode field contains the string "netascii", "octet", or "mail" (or any combination of upper and lower case, such as "NETASCII", NetAscii", etc.) in netascii indicating the three modes defined in the protocol. A host which receives netascii mode data must translate the data to its own format.  Octet mode is used to transfer a file that is in the 8-bit format of the machine from which the file is being transferred. It is assumed that each type of machine has a single 8-bit format that is more common, and that that format is chosen. If a host receives a octet file and then returns it, the returned file must be identical to the original. Mail mode uses the name of a mail recipient in place of a file and must begin with a \verb|WRQ|. Otherwise it is identical to netascii mode. The mail recipient string should be of the form "username" or "username@hostname". If the second form is used, it allows the option of mail forwarding by a relay computer.\\

The discussion above assumes that both the sender and recipient are operating in the same mode, but there is no reason that this has to be the case. For example, one might build a storage server. There is no reason that such a machine needs to translate netascii into its own form of text.  Rather, the sender might send files in netascii, but the storage server might simply store them without translation in 8-bit format. Another such situation is a problem that currently exists on DEC-20 systems. Neither netascii nor octet accesses all the bits in a word. One might create a special mode for such a
machine which read all the bits in a word, but in which the receiver stored the information in 8-bit format. When such a file is retrieved from the storage site, it must be restored to its original form to be useful, so the reverse mode must also be implemented. The user site will have to remember some information to achieve this. In both of these examples, the request packets would specify octet mode to the foreign host, but the local host would be in some other mode. No such machine or application specific modes have been specified in TFTP, but one would be compatible with this specification.\\

It is also possible to define other modes for cooperating pairs of hosts, although this must be done with care. There is no requirement that any other hosts implement these. There is no central authority that will define these modes or assign them names.\\
\begin{center}
Figure 5-2: \verb|DATA| packet
\end{center}

Data is actually transferred in \verb|DATA| packets depicted in Figure 5-2. \verb|DATA| packets (\verb|opcode| = 3) have a block number and data field. The block numbers on data packets begin with one and increase by one for each new block of data.  This restriction allows the program to use a single number to discriminate between new packets and duplicates. The data field is from zero to 512 bytes long. If it is 512 bytes long, the block is not the last block of data; if it is from zero to 511 bytes long, it signals the end of the transfer. (See the section on Normal Termination for details.)\\

All  packets other than duplicate \verb|ACK|'s and those used for termination are acknowledged unless a timeout occurs. Sending a \verb|DATA| packet is an acknowledgment for the first \verb|ACK| packet of the previous \verb|DATA| packet. The \verb|WRQ| and \verb|DATA| packets are acknowledged by \verb|ACK| or \verb|ERROR| packets, while \verb|RRQ|
\begin{center}
Figure 5-4: \verb|DATA| packet
\end{center}

and \verb|ACK| packets are acknowledged by \verb|DATA| or  \verb|ERROR| packets. Figure 5-3 depicts an \verb|ACK| packet; the \verb|opcode| is 4. The block number in an \verb|ACK| echoes the block number of the \verb|DATA| packet being acknowledged. A \verb|WRQ| is acknowledged with an \verb|ACK| packet having a block number of zero.
\begin{center}
Figure 5-4: \verb|DATA| packet
\end{center}

An  \verb|ERROR| packet (\verb|opcode| 5) takes the form depicted in Figure 5-4. An  \verb|ERROR| packet can be the acknowledgment of any other type of packet. The  \verb|ERROR| code is an integer indicating the nature of the  \verb|ERROR|. A table of values and meanings is given in the appendix. The  \verb|ERROR| message is intended for human consumption, and should be in netascii. Like all other strings, it is terminated with a zero byte.

\subsection{Normal Termination}
The end of a transfer is marked by a \verb|DATA| packet that contains between 0 and 511 bytes of data (i.e., Datagram length < 516). This packet is acknowledged by an \verb|ACK| packet like all other \verb|DATA| packets. The host acknowledging the final \verb|DATA| packet may terminate its side of the connection on sending the final \verb|ACK|. On the other hand, dallying is encouraged. This means that the host sending the final \verb|ACK| will wait for a while before terminating in order to retransmit the final \verb|ACK| if it has been lost. The acknowledger will know that the \verb|ACK| has been lost if it receives the final \verb|DATA| packet again. The host sending the last \verb|DATA| must retransmit it until the packet is acknowledged or the sending host times out. If the response is an \verb|ACK|, the transmission was completed successfully. If the sender of the data times out and is not prepared to retransmit any more, the transfer may still have been completed successfully, after which the acknowledger or network may have experienced a problem. It is also possible in this case that the transfer was unsuccessful. In any case, the connection has been closed.

\subsection{Premature Termination}
If a request can not be granted, or some error occurs during the transfer, then an \verb|ERROR| packet (\verb|opcode| 5) is sent. This is only a courtesy since it will not be retransmitted or acknowledged, so it may never be received. Timeouts must also be used to detect errors.

\section{Negotiation mechanism}
\subsection{Introduction}
The option negotiation mechanism proposed in this document is a backward-compatible extension to the TFTP protocol. It allows file transfer options to be negotiated prior to the transfer using a mechanism which is consistent with TFTP's Request Packet format. The mechanism is kept simple by enforcing a request-respond-acknowledge sequence, similar to the lock-step approach taken by TFTP itself. While the option negotiation mechanism is general purpose, in that many types of options may be negotiated, it was created to support the Blocksize option.\\

\subsection{Packet Formats}
TFTP options are appended to the \verb|RRQ| and \verb|WRQ| packets. A new type of TFTP packet, the Option Acknowledgment (\verb|OACK|), is used to acknowledge a client's option negotiation request. A new  \verb|ERROR| code, 8, is hereby defined to indicate that a transfer should be terminated due to option negotiation.\\

Options are appended to a TFTP Read Request or Write Request packet as follows:

   +-------+---~~---+---+---~~---+---+---~~---+---+---~~---+---+-->
   | opc |filename| 0 | mode | 0 | opt1 | 0 | value1 | 0 | <
   +-------+---~~---+---+---~~---+---+---~~---+---+---~~---+---+-->

    >-------+---+---~~---+---+
   < optN | 0 | valueN | 0 |
    >-------+---+---~~---+---+
\begin{itemize}
\item \verb|opc|: The \verb|opcode| field contains either a 1, for \verb|RRQ|, or 2, for \verb|WRQ|.
\item \verb|filename|: The name of the file to be read or written. This is a NULL-terminated field.
\item \verb|mode|: The mode of the file transfer: "netascii", "octet", or "mail". This is a NULL-terminated field.
\item \verb|opt1|: The first option, in case-insensitive ASCII (e.g., blksize). This is a NULL-terminated field.
\item \verb|value1|: The value associated with the first option, in case-insensitive ASCII. This is a NULL-terminated field.
\item \verb|optN, valueN|: The final option/value pair. Each NULL-terminated field is specified in case-insensitive ASCII.
\end{itemize}

The options and values are all NULL-terminated, in keeping with the original request format. If multiple options are to be negotiated, they are appended to each other. The order in which options are specified is not significant. The maximum size of a request packet is 512 octets.\\

The \verb|OACK| packet has the following format:

   +-------+---~~---+---+---~~---+---+---~~---+---+---~~---+---+
   | opc | opt1 | 0 | value1 | 0 | optN | 0 | valueN | 0 |
   +-------+---~~---+---+---~~---+---+---~~---+---+---~~---+---+
\begin{itemize}
\verb|opc|: The \verb|opcode| field contains a 6, for \verb|OACK|.
\verb|opt1|: The first option acknowledgment, copied from the original request.
\verb|value1|: The acknowledged value associated with the first option. If and how this value may differ from the original request is detailed in the specification for the option.
\verb|optN, valueN|: The final option/value acknowledgment pair.
\end{itemize}

\subsection{Negotiation Protocol}
The client appends options at the end of the Read Request or Write request packet, as shown above. Any number of options may be specified; however, an option may only be specified once. The order of the options is not significant.\\

If the server supports option negotiation, and it recognizes one or more of the options specified in the request packet, the server may respond with an Options Acknowledgment (O\verb|ACK|). Each option the server recognizes, and accepts the value for, is included in the \verb|OACK|. Some options may allow alternate values to be proposed, but this is an option specific feature. The server must not include in the \verb|OACK| any option which had not been specifically requested by the client; that is, only the client may initiate option negotiation. Options which the server does not support should be omitted from the \verb|OACK|; they should not cause an  \verb|ERROR| packet to be generated. If the value of a supported option is invalid, the specification for that option will indicate whether the server should simply omit the option from the \verb|OACK|, respond with an alternate value, or send an \verb|ERROR|
packet, with  \verb|ERROR| code 8, to terminate the transfer.\\

An option not acknowledged by the server must be ignored by the client and server as if it were never requested. If multiple options were requested, the client must use those options which were acknowledged by the server and must not use those options which were not acknowledged by the server.\\

When the client appends options to the end of a Read Request packet, three possible responses may be returned by the server:
\begin{itemize}
\item \verb|OACK| - acknowledge of Read Request and the options;
\item \verb|DATA| - acknowledge of Read Request, but not the options;
\item \verb|ERROR| - the request has been denied.
\end{itemize}

When the client appends options to the end of a Write Request packet, three possible responses may be returned by the server:
\begin{itemize}
\item \verb|OACK| - acknowledge of Write Request and the options;
\item \verb|ACK|  - acknowledge of Write Request, but not the options;
\item \verb|ERROR| - the request has been denied.
\end{itemize}

If a server implementation does not support option negotiation, it will likely ignore any options appended to the client's request. In this case, the server will return a \verb|DATA| packet for a Read Request and an \verb|ACK| packet for a Write Request establishing normal TFTP data transfer. In the event that a server returns an  \verb|ERROR| for a request which carries an option, the client may attempt to repeat the request without appending any options. This implementation option would handle servers which consider extraneous data in the request packet to be erroneous.\\

Depending on the original transfer request there are two ways for a client to confirm acceptance of a server's \verb|OACK|. If the transfer was initiated with a Read Request, then an \verb|ACK| (with the data block number set to 0) is sent by the client to confirm the values in the server's \verb|OACK| packet. If the transfer was initiated with a Write Request, then the client begins the transfer with the first \verb|DATA| packet, using the negotiated values. If the client rejects the \verb|OACK|, then it sends an  \verb|ERROR| packet, with  \verb|ERROR| code 8, to the server and the transfer is terminated.\\

Once a client acknowledges an O\verb|ACK|, with an appropriate non-\verb|ERROR| response, that client has agreed to use only the options and values
returned by the server. Remember that the server cannot request an option; it can only respond to them. If the client receives an \verb|OACK| containing an unrequested option, it should respond with an  \verb|ERROR| packet, with  \verb|ERROR| code 8, and terminate the transfer.

Examples:
\begin{itemize}
\item Read Request

   client                      server
   -------------------------------------------------------
   |1|foofile|0|octet|0|blksize|0|1432|0| -->        \verb|RRQ|
                  <-- |6|blksize|0|1432|0|  \verb|OACK|
   |4|0| -->                        \verb|ACK|
               <-- |3|1| 1432 octets of data |  \verb|DATA|
   |4|1| -->                        \verb|ACK|
               <-- |3|2| 1432 octets of data |  \verb|DATA|
   |4|2| -->                        \verb|ACK|
               <-- |3|3|<1432 octets of data |  \verb|DATA|
   |4|3| -->                        \verb|ACK|

\item Write Request

   client                      server
   -------------------------------------------------------
   |2|barfile|0|octet|0|blksize|0|2048|0| -->        \verb|RRQ|
                  <-- |6|blksize|0|2048|0|  O\verb|ACK|
   |3|1| 2048 octets of data | -->             \verb|DATA|
                          <-- |4|1|  \verb|ACK|
   |3|2| 2048 octets of data | -->             \verb|DATA|
                          <-- |4|2|  \verb|ACK|
   |3|3|<2048 octets of data | -->             \verb|DATA|
                           <-- |4|3|  \verb|ACK|
\end{itemize}

\subsection{Blocksize Option Specification}
The TFTP Read Request or Write Request packet is modified to include the blocksize option as follows. Note that all fields except "opc" are NULL-terminated.

   +-------+---~~---+---+---~~---+---+---~~---+---+---~~---+---+
   | opc |filename| 0 | mode | 0 | blksize| 0 | \# octets| 0 |
   +-------+---~~---+---+---~~---+---+---~~---+---+---~~---+---+

\begin{itemize}
\item \verb|opc|: The \verb|opcode| field contains either a 1, for Read Requests, or 2, for Write Requests.
\item \verb|filename|: The name of the file to be read or written.
\item \verb|mode|: The mode of the file transfer: "netascii", "octet", or "mail".
\item \verb|blksize|: The Blocksize option, "blksize" (case in-sensitive).
\item \verb|\# octets|: The number of octets in a block, specified in ASCII. Valid values range between "8" and "65464" octets, inclusive. The blocksize refers to the number of data octets; it does not include the four octets of TFTP header.
\end{itemize}

For example:

   +-------+--------+---+--------+---+--------+---+--------+---+
   |  1  | foobar | 0 | octet | 0 | blksize| 0 | 1428 | 0 |
   +-------+--------+---+--------+---+--------+---+--------+---+

is a Read Request, for the file named "foobar", in octet (binary) transfer mode, with a block size of 1428 octets (Ethernet MTU, less the TFTP, UDP and IP header lengths). If the server is willing to accept the blocksize option, it sends an Option Acknowledgment (\verb|OACK|) to the client. The specified value must be less than or equal to the value specified by the client. The client must then either use the size specified in the \verb|OACK|, or send an \verb|ERROR| packet, with  \verb|ERROR| code 8, to terminate the transfer. The reception of a data packet with a data length less than the negotiated blocksize is the final packet. If the blocksize is greater than the amount of data to be transfered, the first packet is the final packet. If the amount of data to be transfered is an integral multiple of the blocksize, an extra data packet containing no data is sent to end the transfer.

\subsection{Timeout Interval Option Specification}
The TFTP Read Request or Write Request packet is modified to include the timeout option as follows:

   +-------+---~~---+---+---~~---+---+---~~---+---+---~~---+---+
   | opc |filename| 0 | mode | 0 | timeout| 0 | \# secs | 0 |
   +-------+---~~---+---+---~~---+---+---~~---+---+---~~---+---+

\begin{itemize}
\item \verb|opc|: The \verb|opcode| field contains either a 1, for Read Requests, or 2, for Write Requests
\item \verb|filename|: The name of the file to be read or written. This is a NULL-terminated field.
\item \verb|mode|: The mode of the file transfer: "netascii", "octet", or "mail". This is a NULL-terminated field.
\item \verb|timeout|: The Timeout Interval option, "timeout" (case in-sensitive). This is a NULL-terminated field.
\item \verb|\# secs|: The number of seconds to wait before retransmitting, specified in ASCII. Valid values range between "1" and "255" seconds, inclusive. This is a NULL-terminated field.
\end{itemize}

For example:

   +-------+--------+---+--------+---+--------+---+-------+---+
   |  1  | foobar | 0 | octet | 0 | timeout| 0 |  1  | 0 |
   +-------+--------+---+--------+---+--------+---+-------+---+

is a Read Request, for the file named "foobar", in octet (binary) transfer mode, with a timeout interval of 1 second.\\

If the server is willing to accept the timeout option, it sends an Option Acknowledgment (\verb|OACK|) to the client. The specified timeout value must match the value specified by the client.

\subsection{Transfer Size Option Specification}
The TFTP Read Request or Write Request packet is modified to include the tsize option as follows:

   +-------+---~~---+---+---~~---+---+---~~---+---+---~~---+---+
   | opc |filename| 0 | mode | 0 | tsize | 0 | size | 0 |
   +-------+---~~---+---+---~~---+---+---~~---+---+---~~---+---+
\begin{itemize}
\item \verb|opc|: The \verb|opcode| field contains either a 1, for Read Requests, or 2, for Write Requests.
\item \verb|filename|: The name of the file to be read or written. This is a NULL-terminated field.
\item \verb|mode|: The mode of the file transfer: "netascii", "octet", or "mail". This is a NULL-terminated field.
\item \verb|tsize|: The Transfer Size option, "tsize" (case in-sensitive). This is a NULL-terminated field.
\item \verb|size|: The size of the file to be transfered. This is a NULL-terminated field.
\end{itemize}

For example:

   +-------+--------+---+--------+---+--------+---+--------+---+
   |  2  | foobar | 0 | octet | 0 | tsize | 0 | 673312 | 0 |
   +-------+--------+---+--------+---+--------+---+--------+---+

is a Write Request, with the 673312-octet file named "foobar", in octet (binary) transfer mode.

In Read Request packets, a size of "0" is specified in the request and the size of the file, in octets, is returned in the \verb|OACK|. If the file is too large for the client to handle, it may abort the transfer with an  \verb|ERROR| packet (\verb|ERROR| code 3). In Write Request packets, the size of the file, in octets, is specified in the request and echoed back in the O\verb|ACK|. If the file is too large for the server to handle, it may abort the transfer with an \verb|ERROR| packet ( \verb|ERROR| code 3).
  
\section{Appendix}
  Order of Headers
  
                           2 bytes
    ----------------------------------------------------------
   | Local Medium | Internet | Datagram | TFTP opcode |
    ----------------------------------------------------------
  
  TFTP Formats
  
   Type  Op \#   Format without header
  
       2 bytes  string  1 byte   string  1 byte
       -----------------------------------------------
   \verb|RRQ|/ | 01/02 | Filename |  0 |  Mode  |  0 |
   \verb|WRQ|  -----------------------------------------------
       2 bytes  2 bytes    n bytes
       ---------------------------------
   \verb|DATA| | 03  |  Block \# |  Data  |
       ---------------------------------
       2 bytes  2 bytes
       -------------------
   \verb|ACK|  | 04  |  Block \# |
       --------------------
       2 bytes 2 bytes    string  1 byte
       ----------------------------------------
    \verb|ERROR| | 05  |  \verb|ERROR|Code |  ErrMsg  |  0 |
       ----------------------------------------
  
  Initial Connection Protocol for reading a file
  
   1. Host A sends a "\verb|RRQ|" to host B with source= A's \verb|TID|,
     destination= 69.
  
   2. Host B sends a "\verb|DATA|" (with block number= 1) to host A with
     source= B's \verb|TID|, destination= A's \verb|TID|.
  
   \verb|ERROR| Codes
  
   Value   Meaning
  
   0     Not defined, see  \verb|ERROR| message (if any).
   1     File not found.
   2     Access violation.
   3     Disk full or allocation exceeded.
   4     Illegal TFTP operation.
   5     Unknown transfer ID.
   6     File already exists.
   7     No such user.
  
  Internet User Datagram Header [2]
  
   (This has been included only for convenience. TFTP need not be
   implemented on top of the Internet User Datagram Protocol.)
  
    Format
  
    0          1          2          3
    0 1 2 3 4 5 6 7 8 9 0 1 2 3 4 5 6 7 8 9 0 1 2 3 4 5 6 7 8 9 0 1
   +-+-+-+-+-+-+-+-+-+-+-+-+-+-+-+-+-+-+-+-+-+-+-+-+-+-+-+-+-+-+-+-+
   |     Source Port     |    Destination Port    |
   +-+-+-+-+-+-+-+-+-+-+-+-+-+-+-+-+-+-+-+-+-+-+-+-+-+-+-+-+-+-+-+-+
   |      Length       |      Checksum      |
   +-+-+-+-+-+-+-+-+-+-+-+-+-+-+-+-+-+-+-+-+-+-+-+-+-+-+-+-+-+-+-+-+
  
  
   Values of Fields
  
  
   Source Port   Picked by originator of packet.
  
   Dest. Port   Picked by destination machine (69 for \verb|RRQ| or \verb|WRQ|).
  
   Length     Number of bytes in UDP packet, including UDP header.
  
   Checksum    Reference 2 describes rules for computing checksum.
           (The implementor of this should be sure that the
           correct algorithm is used here.)
           Field contains zero if unused.
  
   Note: TFTP passes transfer identifiers (\verb|TID|'s) to the Internet User
   Datagram protocol to be used as the source and destination ports.

\end{document}